\documentclass[a4paper,12pt]{article}

\usepackage[utf8]{inputenc}
% \usepackage{showframe}
\usepackage{graphicx}
\usepackage{ragged2e}


\usepackage{geometry}
 \geometry{
 a4paper,
 left=40mm,
 right=30mm,
 top=30mm,
 bottom=30mm,
 }


\title{}
\author{}
\date{}

\pdfinfo{%
  /Title    (Proposal Tugas Akhir)
  /Author   (Achmadi)
  /Creator  (Achmadi)
  /Producer (Achmadi)
  /Subject  (Buat TA)
  /Keywords (TA dan TA)
}

\begin{document}

\pagenumbering{gobble}

\begin{center}
  \textbf{ \Large{PROPOSAL TUGAS AKHIR } }\\[5pt]
  \textbf{ \large{Rancang Bangun \textit{Robot Vision} Berbasis \textit{Color Tracking} Menggunakan OpenCV pada perangkat RaspberryPi } }
  \\[50pt]
  \includegraphics[width=200pt]{ITS}
  \\[50pt]
  \textbf{ \large{Disusun Oleh:} }\\
  \underline{\textbf{ \large{Achmadi} } }\\
  \textbf{ \large{NRP: 2410100085 } }\\[40pt]
  \textbf{ \large{Pembimbing: } }\\
  \textbf{ \large{Bu Apri \hspace{50pt} NIP: } }\\[60pt]
  \textbf{ \large{PROGRAM STUDI S-1 TEKNIK FISIKA } }\\
  \textbf{ \large{JURUSAN TEKNIK FISIKA } }\\
  \textbf{ \large{FAKULTAS TEKNOLOGI INDUSTRI } }\\
  \textbf{ \large{INSTITUT TEKNOLOGI SEPULUH NOPEMBER } }\\
  \textbf{ \large{SURABAYA } }\\
  \textbf{ \large{2014 } }\\
\end{center}
\newpage
\begin{center}
  \textbf{ LEMBAR PENGESAHAN  }\\[5pt]
  \textbf{ PROPOSAL TUGAS AKHIR  } \\[5pt]
  \textbf{ JURUSAN TEKNIK FISIKA FTI-ITS  }\\[5pt]
\end{center}
\rule{385pt}{5pt}
\begin{flushleft}
  \noindent Judul \hspace{80pt} : Rancang Bangun \textit{Robot Vision} Berbasis \textit{Color} \\[5pt]
  \textit{Tracking} Menggunakan OpenCV pada perangkat RaspberryPi \\[5pt]
  \noindent Bidang Studi \hspace{42pt} : Rekayasa Fotonika\\[5pt]
  \noindent 1. Nama Mahasiswa \hspace{5pt} : Achmadi\\[5pt]
  \noindent 2. NRP  \hspace{70pt} : 2410100085\\[5pt]
  \noindent 3. Jangka Waktu \hspace{20pt} : 4 bulan (September-Januari)\\[5pt]
  \noindent 4. Pembimbing \hspace{28pt} : Bu Apri\\[5pt]
  \noindent 5. Usulan ke \hspace{40pt} : 1\\[5pt]
  \noindent 6. Status \hspace{60pt} : Baru\\[5pt]
\end{flushleft}
\rule{385pt}{5pt}
\begin{flushright}
  Surabaya, 15 September 2014
\end{flushright}
\begin{center}
 Pengusul,
 \\[40pt]
 \underline{Achmadi}\\
 NRP: 2410100085
 \\[40pt]
 Menyetujui:
 \\[100pt]
 Mengetahui:\\
\end{center}
\newpage
\begin{center}
 \textbf{PROPOSAL TUGAS AKHIR}
\end{center}
\begin{flushleft}
  \noindent \textbf{I. \hspace{10pt} Judul}\\
  Rancang Bangun \textit{Robot Vision} Berbasis \textit{Color Tracking} Menggunakan OpenCV pada perangkat RaspberryPi
  \\[10pt]
  \noindent \textbf{II. \hspace{9pt} Mata Kuliah Pilihan Bidang Minat Yang Diambil}\\
  1. Pengolahan Citra\\
  2. Metrologi Optik
  \\[10pt]
  \noindent \textbf{III. \hspace{8pt} Pembimbing}\\
  1. Bu Apri
  \\[10pt]
  \noindent \textbf{IV. \hspace{9pt} Latar Belakang}
  \justify \hspace{20pt} Era modern ini tidak akan terlepas dari perkembangan teknologi robot.
  Tujuan utama setiap perancangan robot tentu adalah untuk mengganti pekerjaan manusia.
  Kemampuan robot untuk melakukan pekerjaan yang berulang dan berbahaya telah menjadi
  kebutuhan di setiap lingkungan industri. Untuk memenuhi kebutuhan tersebut telah
  dikembangkan beragam teknologi sensor dan actuator. Khusus untuk sensor, telah
  dikembangkan teknologi yang mirip dengan cara kerja pada indra manusia. Salah satu yang
  banyak dipakai adalah penggunaan kamera sebagai pengganti mata untuk robot.
\end{flushleft}
\end{document}
