\documentclass{article}

% \usepackage{showframe}
\usepackage[utf8]{inputenc}
\usepackage[T1]{fontenc}
\usepackage{mathptmx}
\usepackage{geometry}
 \geometry{
 a5paper,
 left=25mm,
 right=15mm,
 top=25mm,
 bottom=15mm, 
 }
 
% \title{}
% \author{}
% \date{}
% 
% \pdfinfo{%
%   /Title    ()
%   /Author   ()
%   /Creator  ()
%   /Producer ()
%   /Subject  ()
%   /Keywords ()
% }

\renewcommand{\baselinestretch}{1.0}
\makeatletter
\renewcommand\normalsize{\@setfontsize\normalsize{11}{15}}
\renewcommand\large{\@setfontsize\large{13}{15}}
\makeatother

\begin{document}

\begin{center}
 \large{\textbf{BAB I}}\\
 \large{\textbf{PENDAHULUAN}}\\
\end{center}
\vspace{2pt}

\noindent \textbf{1.1 Latar Belakang}

Robot vision adalah robot yang mampu menggunakan kamera sebagai sumber informasi untuk diolah sesuai kebutuhan (Browning,2013).
Tujuan utama setiap perancangan robot tentu adalah untuk mengganti pekerjaan manusia.
Kemampuan robot untuk melakukan pekerjaan yang berulang dan berbahaya telah menjadi kebutuhan di setiap lingkungan industri (Chao,2014).
Untuk memenuhi kebutuhan tersebut telah dikembangkan beragam teknologi sensor dan actuator.
Khusus untuk sensor, telah dikembangkan teknologi yang mirip dengan cara kerja pada indra manusia.
Salah satu yang banyak dipakai adalah penggunaan kamera sebagai pengganti mata untuk robot (Browning,2013).

Penggunaan kamera sebagai sensor visual telah banyak di terapkan pada bidang robotika (Browning,2013).
Sebagai contoh robot Asimo dengan harga \$2.500.000 per unit menggunakan kamera CCD diproses dengan 20 CPU buatan Honda (Honda,2007).
Kemudian robot Noa dengan harga \$7.900 per unit menggunakan kamera SOC yang diproses dengan Intel Atom (Aldebaran,2012).
Sedangkan robot Darwin-Op dengan harga \$12.000 per unit menggunakan kamera CMOS diproses dengan Intel Atom (Robotis,2012).
Berdasarkan informasi tersebut, maka harga tersebut masih tergolong mahal untuk aplikasi skala rumah dan kantor secara umum.

Selain kamera terdapat jenis-jenis sensor lain.
Salah satu contohnya adalah jenis sensor proximity (kedekatan).
Sensor proximity yang umum dipakai adalah sensor berbasis ultrasonik dan IR (Widodo, 2007).
Kedua jenis sensor ini mampu mengukur jarak dekat antara sensor dengan sebuah objek/penghalang (obstacle).
Karena sensor ini hanya mampu memberikan jarang maka sensor ini bagus untuk menghindar (avoiding), namun kurang untuk menemukan objek dengan kriteria tertentu.(Widodo,2007)

RaspberryPi merupakan salah satu jenis SBC (Single Board Computer) dengan spesifikasi processor BCM235 arsitektur arm1176 kategori armhf dengan clock 700MHz dan RAM 512.
Tersedia Operating System yang dapat dijalankan oleh RaspberryPi yaitu Raspbian yang berbasis Linux dan memiliki pustaka pengolahan citra OpenCV di lumbung perangkat lunaknya.
RaspberryPi dan Operating System Raspbian merupakan proyek opensource yang bersifat gratis (Upton,2012).
Perangkat RaspberryPi tersedia dengan harga cukup murah yaitu hanya \$60 dan memiliki antar muka USB sehingga dapat berkomunikasi dengan kamera webcam.

Selain RaspberryPi, terdapat banyak SBC lain.
Sebagai contoh adalah pcDuino, SBC yang bentuk fisiknya mirip dengan arduino.
SBC pcDuino memiliki core ARM Cortex A8 dengan clock 1GHz dan DRAM 1GB.
Melihat dari spesifikasi dapat terlihat bahwa pcDuino memiliki kemampuan lebih tinggi.
Namun dari sisi harga, pcDuino relatif dua kali lipat dibandingkan dengan RaspberryPi.
Selain itu, pcDuino membutuhkan arus 2000mA sehingga lebih boros tenaga dan lebih membutuhkan sistem pendinginan di CPU ketimbang RaspberryPi.
Dengan pertimbangan kemampuan minimal yang dibutuhkan dengan harga dan konsumsi tenaga maka dipilihlah RaspberryPi.

Pustaka OpenCV (Open Computer Vision) merupakan pustaka pemrograman berbasis C/C++/Python yang berisi fungsi-fungsi untuk akuisisi dan pengolahan citra.
Pustaka OpenCV juga merupakan proyek opensource yang bersifat gratis.
Saat ini pustaka OpenCV telah diterapkan di banyak website dan aplikasi mobile untuk deteksi wajah dan penjejak warna (OpenCV,2014).

Penjejakan warna (Color Tracking) merupakan salah satu bentuk pengenalan objek yang cukup sederhana. 
Prinsip proses penjejakan warna adalah mensegmentasi gambar sesuai warna kemudian membaca perubahan koordinat warna yang tidak tersegmentasi relatif terhadap dimensi gambar.
Prinsip penjejakan warna ini dapat digunakan untuk mengenali suatu objek dan membuang objek lain sesuai warna kemudian mengikuti perubahan pixel warna yang tidak tersegmentasi untuk mengetahui perubahan posisi objek yang dikenali.

Selain metode penjejakan warna, dikenal pula metode lain seperti penjejakan berbasis kontur.
Dengan metode ini, gambar diolah berdasarkan pixel yang sama untuk mendeteksi garis tepi (Egde Detection).
Setelah di dapat pola garis tepi maka kemudian di cari pendekatan geometri garis tepi dengan persamaan matematika tertentu.
Dibandingkan dengan penjejakan warna, penjejakan berbasis kontur tergolong lebih rumit karena memerlukan persamaan matematika untuk dapat mengenali pola sehingga mwembutuhkan banyak macam persamaan untuk bentuk yang berbeda-beda (Hermawati,2013).
\\
\vspace{2pt}

\noindent \textbf{1.2 Perumusan Masalah}

Permasalahan dalam tugas akhir ini adalah bagaimana merancang bangun robot vision berbasis color tracking pada perangkat RaspberryPi.
\\
\vspace{2pt}

\noindent \textbf{1.3 Batasan Masalah}

Untuk mencegah meluasnya permasalahan maka ditentukanlah batasan masalah yang digunakan dalam tugas akhir ini yaitu:
\begin{enumerate}
 \item Warna objek yang menjadi bahan uji adalah warna yang ditentukan dalam algoritma robot.
 \item Penjejakan objek hanya berdasarkan warna, bukan berdasarkan bentuk atau kontur.
 \item Algoritma robot hanya untuk mencari satu jenis warna, bukan dua atau lebih warna dalam satu waktu.
 \item Algoritma robot tidak dirancang untuk menjejak lebih dari satu objek.
 \item Robot bekerja pada tingkat pencahayaan yang relatif konstan.
 \item Parameter fisis lain selain tingkat pencahayaan diabaikan.
 \item Perangkat Raspberry Pi dijalankan secara default tanpa overclock maupun modifikasi hardware lainnya.
\end{enumerate}

\vspace{2pt}

\noindent \textbf{1.4 Tujuan Tugas Akhir}

Tujuan dari tugas akhir ini adalah merancang bangun robot vision berbasis color tracking pada perangkat RaspberryPi.
\\
\vspace{2pt}

\noindent \textbf{1.4 Manfaat Penelitian}

Motivasi dilakukan penelitian ini adalah penelitian ini mempunyai prospek ke depan yang bermanfaat baik dalam segi aplikasi.
Manfaat dari penelitian ini adalah hasil rancangan dasar sistem navigasi robot yang bersifat visual dengan biaya pengembangan yang relatif murah.
Karena pada dasarnya rancangan ini hanya menggunakan web-kamera dan mini-komputer maka pengembangan rancangan ini tergolong mudah.
Dengan rancangan dasar ini diharapakan di masa depan di setiap rumah tangga dapat membangun sendiri robot semacam ini untuk kebutuhan rumah tangga.
Tidak menutup kemungkinan juga pengunaan di wilayah militer maupun perusahaan besar.
\\
\vspace{2pt}

\noindent \textbf{1.6 Sistematika Laporan}

Laporan dari tugas akhir ini akan disusun dalam 5 bab sebagai berikut:

\begin{enumerate}
 \item Bab I Pendahuluan\\
 Pendahuluan berisi tentang latar belakang masalah yang diambil, perumusan masalah, batasan masalah, tujuan penelitian, manfaat penelitian, dan sistematika laporan.
 
 \item Bab II Dasar Teori\\
 Dasar teori berisi tentang teori-teori yang terkait dengan tugas akhir ini.

 \item Bab III Metode Penelitian\\
 Metode penelitian berisi tentang penjelasan mengenai desain robot dan desain pengujian.
 
 \item Bab IV Analisa Data dan Pembahasan\\
 Analisa data dan pembahasan berisi mengenai hasil eksperimen yang dilakukan, analisis dari hasil yang didapatkan, dan pembahasan.
 
 \item Bab V Penutup\\
 Penutup berisi tentang kesimpulan yang didapatkan dari hasil penelitian dan saran untuk melakukan penelitian selanjutnya.\\

\end{enumerate}

\newpage
.\\
\vspace{200pt}

\hspace{75pt} Halaman ini memang kosong

\end{document}
